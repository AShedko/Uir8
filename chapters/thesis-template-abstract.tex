\chapter*{Реферат}
\thispagestyle{plain}

Пояснительная записка содержит 43 страницы (из них 6 страниц приложений).   Количество использованных источников --- 23. Количество приложений --- 2. Количество иллюстраций --- 16.

Ключевые слова: визуальная локализация, геолокация, глубокое обучение, нейронные сети, LSTM, свёрточные сети

Целью данной работы является разработка системы визуальной локализации на основе глубокого  

В первой главе проводится обзор и анализ геолокации по изображениям, исторических решений задачи визуальной локализации,
алгоритмов разбиения земной поверхности, существубщих библиотек для глубокого обучения.

Во второй главе описываются использованные и модифицированные алгоритмы классификации, архитектуры нейросетей, способы оценки точности классификации, функции потерь.

В третьей главе приводится описание программной реализации.

В четвёртой главе описаны результаты экспериментальной проверки, 
приводятся иллюстрации, описания результатов.

В приложении \ref{app-format} находятся фргменты исходных текстов разработанного ПО

В приложении \ref{morefigs} находятся дополнительные иллюстрации.