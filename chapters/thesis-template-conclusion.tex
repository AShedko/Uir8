\chapter*{Заключение}
\addcontentsline{toc}{chapter}{Заключение}

\begin{itemize}
	\item были проанализированы методы геолокации по изображениям, исторические решения задачи визуальной локализации, алгоритмы разбиения земной поверхности, существубщих библиотек для глубокого обучения, подход переноса параметров.
	
	В итоге были выбраны библиотеки и модели/алгоритмы для дальнейшего изучения.
	
	\item были описаны использованные и модифицированные алгоритмы классификации, архитектуры нейросетей, способы оценки точности классификации, функции потерь;
	
	\item был спроектирована система для геолокации серий изображений;
	
	\item были проведены испытания разработанного прототипа в рамках подзадачи общей задачи геолокации.
\end{itemize}

Была использована библиотека \texttt{pytorch} для упрощения работы с различными архитектурами.

Было выяснено что для задачи многклассовой классификации дообучение всей сети приводит лишь
к переобучению к тестовой выборке, так как результат на валидационной выборке и $ F1 $-мера
для случая переобучения последнего слоя близки к случаю дообучения всех параметров сети.

На модельном датасете получены результаты, сопоставимые с SOTA в данной области:\\
$ F_1 = 0.452 $\\
$ \mbox{Точность} = 0,447 $\\
значительно превосходящие вероятность случайного угадывания.

Направления будущей работы:

\begin{itemize}
	\item В рамках практимки улучшить точность классификации и увеличить её охват
	
	\item Описать более подробно работу с последовательностями
	
	\item Рассмотреть возможность применения подобной архитектуры к другим задачам машинного обучения/ компьютерного зрения
	
	\item Визуализировать состояние обученных слоёв сети.
\end{itemize}