\chapter*{Введение}
\label{sec:afterwords}
\addcontentsline{toc}{chapter}{Введение}

Задача геолокации по изображениям имеет множество приложений в различных сферах человеческой деятельности. На различных масштабах это могут быть: 
\begin{itemize}
\item ориентация в помещениях,
\item автоматизированное определение местоположения по фотографии без геотегов,
\item помощь в ориентации в незнакомой среде.
\end{itemize}

Актуальность данной работы проявляется в:
	\begin{itemize}
		\item Проблематика данной области интересна в широком круге применений, таких как: 
		поиск расположения определённого объекта, автоматическое отслеживание преступника по фотографии с места проишествия , теоретические исследования алгоритмов глубокого обучения для многоклассовой пространственной классификации. Выиграют от решения этой проюлемы специальные службы и корпорации, занимающиеся подобными вещами.
		\item Задача "Визуальной локализации" \cite{im2gps}, поставленная в области
		компьютерного зрения одной из первых оказалась непростой как для компьютеров,
		так и для людей. Первой была работа У. Б. Томпосна (Университет Юты) и других \cite{thompson1999geomReas}, основанная на распознавании деталей ландшафта и сопоставлении их с картой местности.
		Однако при наличии объёмных обучающих выборок ситуация значительно улучшается. Современные работы Т.Вейанда (Google) и И.Кострикова (RWTH Aachen) \cite{weyand2016planet} показывает значительные успехи в данной области.
			
	\end{itemize}

	Все предыдущие работы используют значительные объёмы данных и вычислительные мощности, а потому интересна задача решения подобных задач на менее производительном железе, что может быть полезно для развёртывания этой  технологии в рамках мобильного приложения.

	Новизна работы состоит в использовании для решения задачи современных методов машинного обучения, применения нового обучающего набора и методе его получения.
	
	Суть исследования --- описание и разработка алгоритма геолокации на основе набора изображений.

