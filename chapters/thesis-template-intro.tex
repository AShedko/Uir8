\chapter*{Введение}
\label{sec:afterwords}
\addcontentsline{toc}{chapter}{Введение}

Введение всегда содержит краткую характеристику работы по следующим аспектам:

\begin{itemize}
	\item актуальность:
	\begin{itemize}
		\item Проблематика данной области интересна в широком круге применений, таких как: 
		поиск расположения определённого объекта, автоматическое отслеживание преступника по фотографии с места проишествия , теоретические исследования алгоритмов глубокого обучения для многоклассовой пространственной классификации. Выиграют от решения этой проюлемы специальные службы и корпорации, занимающиеся подобными вещами.
		\item Задача "Визуальной локализации" \cite{im2gps}
		краткая история вопроса (в формате год-фамилия-что сделал),
		\item нерешенные вопросы/проблемы;
	\end{itemize}
	\item новизна работы (что нового привносится данной работой);
	\item оригинальная суть исследования;
	\item содержание по главам (по одному абзацу на главу).
\end{itemize}

Общий объем введения должен не превышать 1,5 страниц (для ПЗ к УИРам может быть чуть меньше).


