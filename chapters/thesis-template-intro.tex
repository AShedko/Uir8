\chapter*{Введение}
\label{sec:afterwords}
\addcontentsline{toc}{chapter}{Введение}

Введение всегда содержит краткую характеристику работы по следующим аспектам:

\begin{itemize}
	\item актуальность работы проявляется в:
	\begin{itemize}
		\item Проблематика данной области интересна в широком круге применений, таких как: 
		поиск расположения определённого объекта, автоматическое отслеживание преступника по фотографии с места проишествия , теоретические исследования алгоритмов глубокого обучения для многоклассовой пространственной классификации. Выиграют от решения этой проюлемы специальные службы и корпорации, занимающиеся подобными вещами.
		\item Задача "Визуальной локализации" \cite{im2gps}, поставленная в компьютерном зрении одной из первых оказалась непростой как для компьютеров, так и для людей \cite{Thompson1999geomReas}. Однако при наличии объёмных обучающих выборок ситуация значительно улучшается. Современные работы Т.Вейанда (Google) и И.Кострикова (RWTH Aachen) \cite{weyand2016planet} показывает значительные успехи в данной области.
		
		Все предыдущие работы используют значительные объёмы данных и вычислительные мощности, а потому интересна задача решения подобных задач на менее производительном железе, что может быть полезно для развёртывания этой  технологии в рамках мобильного приложения.
	\end{itemize}
	\item новизна работы состоит в использовании для решения задачи современных методов машинного обучения и применения нового обучающего набора.
	\item оригинальная суть исследования --- описание и разработка алгоритма геолокации
	\item содержание по главам (по одному абзацу на главу).
\end{itemize}

Общий объем введения должен не превышать 1,5 страниц (для ПЗ к УИРам может быть чуть меньше).


